%Pakete;
%A4, Report, 12pt
\documentclass[ngerman,a4paper,12pt]{scrreprt}
\usepackage[a4paper, right=20mm, left=20mm,top=30mm, bottom=30mm, marginparsep=5mm, marginparwidth=5mm, headheight=7mm, headsep=15mm,footskip=15mm]{geometry}

%Papierausrichtungen
\usepackage{pdflscape}
\usepackage{lscape}

%Deutsche Umlaute, Schriftart, Deutsche Bezeichnungen
\usepackage[utf8]{inputenc}
\usepackage[T1]{fontenc}
\usepackage[ngerman]{babel}

%quellcode
\usepackage{listings}

%tabellen
\usepackage{tabularx}

%listen und aufzählungen
\usepackage{paralist}

%farben
\usepackage[svgnames,table,hyperref]{xcolor}

%symbole
\usepackage{latexsym,textcomp}
\usepackage{amssymb}

%font
\usepackage{helvet}
\renewcommand{\familydefault}{\sfdefault}

%durch- und unterstreichen
\usepackage{ulem}

%Abkürzungsverzeichnisse
\usepackage[printonlyused]{acronym}

%Bilder
\usepackage{graphicx} %Bilder
\usepackage{float}	  %"Floating" Objects, Bilder, Tabellen...
\usepackage[space]{grffile} %Leerzechen Problem bei includegraphics
\usepackage{wallpaper} %Seitenhintergrund setzen
\usepackage{transparent} %Transparenz

%Tikz, Mindmaps, Trees
\usepackage{tikz}
\usetikzlibrary{mindmap,trees}
\usepackage{verbatim}

%for
\usepackage{forloop}
\usepackage{ifthen}

%Dokumenteigenschaften
\title{Summary AI}
\author{Tobias Blaser}
\date{\today{}, Uster}


%Kopf- /Fusszeile
\usepackage{fancyhdr}
\usepackage{lastpage}

\pagestyle{fancy}
	\fancyhf{} %alle Kopf- und Fußzeilenfelder bereinigen
	\renewcommand{\headrulewidth}{0pt} %obere Trennlinie
	\fancyfoot[L]{\jobname} %Fusszeile links
	\fancyfoot[C]{Seite \thepage/\pageref{LastPage}} %Fusszeile mitte
	\fancyfoot[R]{\today{}} %Fusszeile rechts
	\renewcommand{\footrulewidth}{0.4pt} %untere Trennlinie

%Kopf-/ Fusszeile auf chapter page
\fancypagestyle{plain} {
	\fancyhf{} %alle Kopf- und Fußzeilenfelder bereinigen
	\renewcommand{\headrulewidth}{0pt} %obere Trennlinie
	\fancyfoot[L]{\jobname} %Fusszeile links
	\fancyfoot[C]{Seite \thepage/\pageref{LastPage}} %Fusszeile mitte
	\fancyfoot[R]{\today{}} %Fusszeile rechts
	\renewcommand{\footrulewidth}{0.4pt} %untere Trennlinie
}

\usepackage{changepage}

% Abkürzungen für Kapitel, Titel und Listen
\input{commands/shortcutsListAndChapter}
\input{commands/TextStructuringBoxes}

%links, verlinktes Inhaltsverzeichnis, PDF Inhaltsverzeichnis
\usepackage[bookmarks=true,
bookmarksopen=true,
bookmarksnumbered=true,
breaklinks=true,
colorlinks=true,
linkcolor=black,
anchorcolor=black,
citecolor=black,
filecolor=black,
menucolor=black,
pagecolor=black,
urlcolor=black
]{hyperref} % Paket muss unbedingt als letzes eingebunden werden!

\usepackage{graphicx}
\begin{document}

% Inhaltsverzeichnis
\tableofcontents
\clearpage

\ch{Einführung Künstliche Intelligenz}

\expl{Periodische Dezimalzahl}{Lässt sich immer durch einen Bruch darstellen \newline
$q=\frac{z_{1} \in \mathbb{Z}}{z_{2} \in \mathbb{Z}\\\{0\}}$}

\definition{Künstliche Intelligenz}{Ist immer ein Modell der Wirklichkeit und damit eine grosse Optimierungsaufgabe. \\ $Objekt \rightarrow Modell \rightarrow Simulation (+Rauschen)$}


\definition{Antinomie}{Konflikt, der nicht lösbar ist, bsp. Barbier, der nur Leuten den Bart schneidet, die sich selbst nicht den Bart schneiden \ra Konflikt mit sich selbst}

\img{img/v2.1.jpg}{}{0.75}{}
\expl{Lernen}{Modelloptimierung durch finden optimaler Parameter.}

\ul
	\li Dame oder Schach erste Anwendungen von KI \ra finden einer optimalen Strategie.
	\li Robotik
	\li Expertensysteme (Nachbildung von Spezialwissen)
	\li Maschinelles Lernen (Konstruktion neues Wissen anhand Vorhandenem)
	\li Computerbeweise (Herleitung / Verifizierung mathematischer Formeln)
	\li Automatisches Programmieren
	\li Spracherkennung
\ulE

\expl{Aspekte der KI}{Information und Algorithmen (Verändern der geg. Informationen)}

\ch{Modellbildung}
\expl{Berechnen}{Beim Berechnen wird immer Information vernichtet \ra PC wird heiss, Energie von Bits wird ``vernichtet''.}
Modell verformt die Wirklichkeit \ra desto einfach das Modell, desto stärker die Verformung

\expl{Satz von Gödel}{Systeme, die versuchen alles zu beschreiben (axiomatisches System) führen zu einem Wiederspruch in sich selbst.}

\expl{Spiel}{Ein Spiel ist bereits ein Modell und findet in unseren Köpfen statt.}


\se{Graphische Methoden und Spieltheorie}
\img{img/v2.2.jpg}{Gerichteter Graph mit Sprague-Grundy Bewertung}{0.25}{}

\img{img/v2.3.jpg}{Rangverteilung für die Spiele Aa und Ab (links) und Rangverteilung für die Spiele
Ba und Bb (rechts)}{0.75}{}

\sse{NIM Summe}
\definition{NIM Summe}{Die NIM-Summe von zwei nat"urlichen Zahlen ist definiert als die bin"are Addition
ohne Mitnahme des Ubertragungsbits (dh. die XOR Bildung). Also: (xm, ..., xo )z(ym, .., yo ) = (zm , .., zo), wobei zi = xi + yi mod (2).}

\img{img/v2.4.jpg}{Wir bezeichnen mit z die NIM-Summe = Sprague-Grundy Rang, welcher zu zwei Zahlen (Spalten- und Zeilennummer a, b ) gehört, also azb = c. \ra Wichtig: a und b beziehen sich auf die Ränge (0...x-1) und nicht Matrixenzeilen/spaltennummer (1...x)}{0.75}{}
 Eigenschaften:
 \ul
 	\li Assoziativität: (az(bzc)) = (azb)zc
 	\li Kommutativität: azb = bza
 	\li 0 als Identität: 0zb = b
 	\li Inversit at: aza = 0
\ulE

5,7,7:
\ol
	\li Ausgangslage (Binäre Zeilennummer/Haufenanzahl): \\
		101 \\
		111 \\
		111 \\
		----- \\
		101
	\li Ich ziehe auf Gewinnerposition (Rang 0): \\
		001 \\
		111 \\
		111 \\
		-----\\
		000
\olE
Zeilen (\#Hölzchenhaufen, Dimensionen bei Gitter) werden xor verknüpft.

\sse{Matrixen}
Jeder Graph kann als Matrix dargestellt werden. \ra Nachbarschaftsmatrix \\
Spalten: von, Zeilen: nach
\ul
	\li quadratische Matrix: nichtnegative Matrix
	\li reguläre Matrix: alle Einträge A\^k sin dpositiv
\ulE

%Skizze s2.1


\ch{Repetition}
Fehler kann nie null sein, aber kann möglichst klein sein.

\expl{Strategie}{Definiert, wie minimaler Fehler erreicht werden kann.}

\expl{Sprague-Grundi-Ränge (Spieltheorie)}{Felder werden mit möglichst kleiner Zahl ausgefüllt, die nicht schon in einem direkten Nachfolgefeld steht. \\ \ra wenn ich auf einem Rang-0 Feld bin, kann der Andere nur auf eine Nicht-0 Zahl ziehen. Danach kann ich entweder auf die Endnull oder ein anderes 0-Feld ziehen. \\ \textbf{\ra Zyklen dürfen nicht vorkommen}}

\expl{NIM-Spiel}{Durch XOR Verknüpfung der Binären Haufenanzahl werden 0-Positionen ermittelt}

\expl{Matrixendarstellung von Graphen}{\ol 
	\li Durchnummerieren der Knoten 
	\li erstellen einer Adjazenzmatrix (Nachbarschaftsmatrix) die n*n Felder besitzt (n=\#Knoten) 
	\li Die Kanten werden als von-zu mit 1 in der Matrix eingetragen 
\olE

Durch multiplizieren eines Punktes mit sich selbst \ra Wenn es eine Einheitsmatrix gibt, ist der Graph nicht zusammenhängend
}
Punkt mit sich selbst multiplizieren (Mathematica):
\begin{verbatim}
A = {{1,0}{1,1}};
A.A
{{1,0}{2,1}};
A.A.A.A.A
\end{verbatim}

\expl{Reguläre Matrix}{Von jedem Punkt aus ist jeder Punkt erreichbar.}


\ch{Lineare Optimierungen}



\ch{Quadratische Optimierungen}




\end{document}
