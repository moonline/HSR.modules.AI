%Pakete;
%A4, Report, 12pt
\documentclass[ngerman,a4paper,12pt]{scrreprt}
\usepackage[a4paper, right=20mm, left=20mm,top=20mm, bottom=30mm, marginparsep=5mm, marginparwidth=5mm, headheight=7mm, headsep=15mm,footskip=15mm]{geometry}

%Papierausrichtungen
\usepackage{pdflscape}
\usepackage{lscape}

%Deutsche Umlaute, Schriftart, Deutsche Bezeichnungen
\usepackage[utf8]{inputenc}
\usepackage[T1]{fontenc}
\usepackage[ngerman]{babel}

%quellcode
\usepackage{listings}

%tabellen
\usepackage{tabularx}

%listen und aufzählungen
\usepackage{paralist}

%farben
\usepackage[svgnames,table,hyperref]{xcolor}

%symbole
\usepackage{latexsym,textcomp}

%font
\usepackage{helvet}
\renewcommand{\familydefault}{\sfdefault}

%Abkürzungsverzeichnisse
\usepackage[printonlyused]{acronym}

%Bilder
\usepackage{graphicx} %Bilder
\usepackage{float}	  %"Floating" Objects, Bilder, Tabellen...
\usepackage[space]{grffile} %Leerzechen Problem bei includegraphics
\usepackage{wallpaper} %Seitenhintergrund setzen
\usepackage{transparent} %Transparenz

%for
\usepackage{forloop}
\usepackage{ifthen}

%Dokumenteigenschaften
\title{Repetitionsfragen CN1}
\author{Tobias Blaser}
\date{\today{}, Rapperswil}


%Kopf- /Fusszeile
\usepackage{fancyhdr}
\usepackage{lastpage}

\pagestyle{fancy}
	\fancyhf{} %alle Kopf- und Fußzeilenfelder bereinigen
	\renewcommand{\headrulewidth}{0pt} %obere Trennlinie
	\fancyfoot[L]{Seite \thepage/\pageref{LastPage}} %Fusszeile mitte
	\fancyfoot[R]{\today{}} %Fusszeile rechts
	\renewcommand{\footrulewidth}{0.4pt} %untere Trennlinie

%Kopf-/ Fusszeile auf chapter page
\fancypagestyle{plain} {
	\fancyhf{} %alle Kopf- und Fußzeilenfelder bereinigen
	\renewcommand{\headrulewidth}{0pt} %obere Trennlinie
	\fancyfoot[L]{Seite \thepage/\pageref{LastPage}} %Fusszeile mitte
	\fancyfoot[R]{\today{}} %Fusszeile rechts
	\renewcommand{\footrulewidth}{0.4pt} %untere Trennlinie
}

\usepackage{changepage}

% Abkürzungen für Kapitel, Titel und Listen
\input{commands/shortcutsListAndChapter}
\input{commands/TextStructuringBoxes}

%links, verlinktes Inhaltsverzeichnis, PDF Inhaltsverzeichnis
\usepackage[bookmarks=true,
bookmarksopen=true,
bookmarksnumbered=true,
breaklinks=true,
colorlinks=true,
linkcolor=black,
anchorcolor=black,
citecolor=black,
filecolor=black,
menucolor=black,
pagecolor=black,
urlcolor=black
]{hyperref} % Paket muss unbedingt als letzes eingebunden werden!

\usepackage{graphicx}
\begin{document}

% Inhaltsverzeichnis
\tableofcontents

\vspace{1cm}
\section*{Antworten zu den Repetitionsfragen}
Falls vorhanden befinden sich diese im GitHub Repository. Ergänzungen oder ganze Antwortensets sind jederzeit herzlich willkommen.

\noindent\url{https://github.com/moonline}
\clearpage

\ch{Analytische Optimierungsprobleme}
\ol
	\li Erklären Sie das Barbier Problem. Ist es lösbar?
\olS

\ch{Einführung in die künstliche Intelligenz}
\olR
	\li Nennen Sie einige Gebiete, in denen künstliche Intelligenz zur Anwendung kommt, oder kommen könnte.
	\li Mit welchen zwei Aspekten beschäftigt sich KI hauptsächlich?
	\li Warum verformt ein Modell die Wirklichkeit?
	\li Was beschreibt der Satz von Gödel?
	\li Erklären Sie das Dodon's Problem. Lösen Sie das Problem mit Mathematica.
	\li Erklären Sie das Problem ``Treib die Dame in die Ecke''. Lösen Sie das Problem mit Mathematica.
	\li Wie funktioniert die Sprague-Grundy Bewertung? Inwiefern trägt diese zur Lösung des zuletzt genannten Problems bei? Warum darf es bei Sprague-Grundy keine Schleifen geben?
\olS

\ch{Matrixen}
\olR
	\li Was ist eine quadratische Matrix?
	\li Was ist eine reguläre Matrix?
	\li Was ist eine Adjazenzmatrix?
	\li Was sind rechts- und links stochiastische Matrixen?
	\li Wie stellen Sie einen Graph als Matrix dar?
	\li Erklären Sie, wie sie durch Matrixmultiplikation ermitteln können, ob ein Graph zusammenhängend ist, oder nicht.
\olS


\ch{Optimierungen}
\se{Eindimensionale Optimierungen}
\olR
	\li Erklären Sie anhand den folgenden Beispielen, wie lineare Optimierungen funktionieren:
		\ol
			\li Zwei Kerzen mit gleicher Leuchtkraft und Abstand a. Wo auf ihrer direkten Verbindung ist die Helligkeit am grössten? Rechnen Sie, Raten Sie nicht!
			\li Von A soll man auf möoglichst schnellem Weg nach B gehen. A(0, 0) liegt auf einer gera-
den horizontalen Strasse, B(a, b) im Feld. Auf der Strasse hat man die (grössere) Geschwindigkeit
v1 als im Feld (v2 ). Wo biegt man von der Strasse ab?
		\olE
	\li Was sind Lagrange Multiplikatoren? Erklären Sie anhand des folgenden Beispiels die Lagrange Multiplikatioren: \\
		Beispiel 11. Zwei Schiffe fahren gleichzeitig los, auf den Seiten eines Dreieckes von A nach B und von B nach C, bzw., mit verschiedenen Geschwindigkeiten. Der Winkel bei B sei b. Wann sind
sie sich am näachsten?
	\li Was sind Nebenbedingungsfaktoren?
\olS

\se{Mehrdimensionale Optimierungen}
\olR
	\li Erklären Sie anhand des folgenden Beispiels, wie mehrdimensionale Optimierungen funktionieren: \\
		Finden Sie das Minimum der Funktion $x^2 + 0.25 y^2$
	\li Erklären Sie den Gradientenabstieg.
	\li Erklären Sie das Simplexverfahren
	\li Was ist eine konvexe Menge?
\olS


\ch{Nichttriviale Lösungsmengen}
\olR
	\li Was sind nichttriviale Lösungsmengen?
	\li Erklären Sie das Chaos-Spiel.
	\li Erstellen Sie für das Chaos Spiel ein Programm mit Mathematica.
	\li Was müssen Sie tun, um mit dem Ansatz des Chaos Spiels dreidimensionale Bilder zu erhalten?
\olS


\ch{Neuronale Netze}
\olp{
	\li Zeichnen Sie ein Neuron und beschriften Sie die wichtigsten drei Teile.
	\li Was ist ein Perzeptron? Wie funktioniert es?
	\li Wie funktioniert das Lernen bei einem Neuron? In welchem Fall wird das Neuron gestärkt?  Was ist der Unterschied zwischen überwachtem Lernen und selbstorganisiertem Lernen?
	\li Was sind die 'Gewichte' bei einem Neuron?
	\li Skizzieren Sie einen 'Feuerstoss' eines Neurons und beschriften Sie die Skizze. Wozu dient die Feuerschwelle?
	\li Was ist die Ausgabefunktion eines Neurons?
	\li Erklären Sie die beiden Arten des Hebschen Lernens.
	\li Warum kann mit einem Neuron keine XOR Verknüpfung gebildet werden? Wie muss das Neuron modifiziert werden, damit auch XOR Verknüpfungen möglich sind?
}

\se{Einfache Netze}



\se{Imaginäre Gewichte}
\olp{
	\li Welchen Vorteil bieten Holographische Perzeptronen?
	\li Können holographische Perzeptronen mehr als normale?
	\li Machen Sie ein Beispiel, wie Sie mit einem Netz von holographischen Neuronen die Wurzelfunktion lernen.
}


\ch{Kohonennetzwerke \& Monte Carlo Funktion}
\se{Kohonennetzwerke}
\olp{
	\li Erklären Sie anhand eines Beispiels, wie ein Kohonennetzwerk funktioniert.
	\li Warum gibt es Schwierigkeiten bei zu grossen Kohonennetzwerken?
	\li Beurteilen Sie das Resultat eines Kohonennetzwerkes. Wie genau sind die gelieferten Resultate? Kann man davon ausgehen, dass man das absolute Minimu gefunden hat?
	\li Was kann man erreichen, wenn der optimale Pfad mehrmals berechnet wird?
}

\se{Monte Carlo}
\olp{
	\li Erklären Sie, wie der Montecarlo Algorithmus funktioniert.
	\li Wie wird ein erreichtes lokales Minimum wieder verlassen, um en weiteres zu finden?
}


\ch{Hopfield Netzwerk}
\olp{
	\li Erklären Sie, nach welchen Prinzip das Hopfield Netzwerk funktioniert.
	\li Wie können Sie mit dem Hopfield Netzwerk Abbildungen zwischen Pattern vergleichen?
}


\ch{Genetische Algorithmen}
\olp{
	\li Erklären Sie die Begriffe Mutation, Kreuzung, Fitness
	\li Erklären Sie das Grundprinzip, wie sie mit einem genetischen Algorithmus ein Minima in einer Fitnessfunktion finden können
	\li Welche Rolle spielt das ``Glücksrad'' bei genetischen Algorithmen?
}

\se{Genetische Programmierung}
\olp{
	\li Erklären Sie, was genetische Programmierung ist.
	\li Nennen Sie einige Bedingungen, die erfüllt sein müssen, damit Sie mittels genetischer Programmierung ein Problem lösen können.
}

\se{Clustering}
\olp{
	\li Was sind Clustering Gruppierverfahren?
	\li Welhem Prinzip müssen Clustering Gruppierverfahren folgen, damit konvexe und konkave Trennungen möglich sind?
	\li Was sind automone Klusteringverfahren?
	\li Erklären Sie K-means
	\li Erklären Sie Ward-Clustering
	\li Erklären Sie SSC
}



\end{document}
